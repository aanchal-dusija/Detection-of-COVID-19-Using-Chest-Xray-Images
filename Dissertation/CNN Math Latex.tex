\begin{document}
CNN Math 

Convolutional Neural Network (CNN) is a robust deep learning algorithm, mainly used to identify and classify images. CNN is a distinctive type of neural network in which analyses visual inputs to identify for segmentation, detection, and classification. The concept is based on combining low-level features in the image to higher and higher-level features. The filters extract features from images in such a way that the position information of pixels is retained. The algorithm is mainly used for face recognition, analysing documents, managing traffic in smart cities, and recommendation systems.
 
CNN is trained on the back-propagation algorithm. After we get the output from Forward propagation, the actual value is compared with the output to calculate the error rate. The parameters are then updated, and the entire process is repeated to get the optimal values.
% (https://www.analyticsvidhya.com/blog/2020/02/mathematics-behind-convolutional-neural-network/?utm_source=blog&utm_source=learn-image-classification-cnn-convolutional-neural-networks-5-datasets)
 
CNN is an amalgamation of biology, art, and mathematics. The biology of the eye inspires the entire architecture. CNN mimics the connectivity of neurons within the brain. While we as humans perceive a visual image as a detailed, coloured image of the world around us, there is quite a lot of processing done in our brain to get to this point.  

CNN has proved particularly successful in working with image data and ever since being used in ImageNet competition in 2012. They have been the frontrunners in research and industry while dealing with images. 
%https://courses.analyticsvidhya.com/courses/convolutional-neural-networks-cnn-from-scratch?utm_source=blog&utm_medium=mathematics-behind-convolutional-neural-network

 
CNN can also be used for deep learning applications in healthcare, such as medical imaging. CNN has been used for features learning on breast ultrasound images, blood analysis, brain lesion segmentation, detection of Alzheimer’s and Parkinson’s Diseases, tumour, lung cancer, pneumonia, and various other diseases. 
 
%(https://arxiv.org/ftp/arxiv/papers/1704/1704.06825.pdf) 


